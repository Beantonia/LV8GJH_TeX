\documentclass[twoside]{article}
\usepackage{hulipsum}
\usepackage[magyar]{babel}
\usepackage{geometry}

\geometry{bindingoffset=0cm}

\geometry{inner=3cm,outer=4cm,top=3cm,bottom=3cm}
\begin{marginpar}

\end{marginpar}

\begin{document}


\begin{abstract}

\hulipsum
\footnote{nem tudom, valami szöveg}
\end{abstract}
\title{Napraforgó}
\author{Cserepes Virág}
\date{2025. július 8.}
\maketitle

\renewcommand{\thefootnote}{\fnsymbol{footnote}}

\section[Virágok]{A kerti virágok élete}
\footnote{nem tudom, valami szöveg}
\subsection{}
\hulipsum
\subsection{}
\hulipsum
\section[A búzavirág]{Búzavirág}
\subsection{1.1}
\subsubsection{1.1.1}
\paragraph{1.1.1.1.1}
\subparagraph{1.1.1.1.1.1}
\tableofcontents
\setcounter{tocdepth}{1.1.1}
\pagenumbering{roman}

\begin{verse}
Ide kéne valami versike,
Meg is írom, gyorsan izibe,
Nem lesz szép vagy különleges,
De erre sok időt szánni felesleges.
\end{verse}

\begin{quote}
valamiféle idézetre lenne szükség
\end{quote}

\begin{quotation}
valamiféle idézetre lenne szükség
\end{quotation}
\section[Virágok]{A kerti virágok élete}
\footnote{nem tudom, valami szöveg}
\subsection{}
\hulipsum
\subsection{}
\hulipsum\section[Virágok]{A kerti virágok élete}
\footnote{nem tudom, valami szöveg}
\subsection{}
\hulipsum
\subsection{}
\hulipsum

\end{document}