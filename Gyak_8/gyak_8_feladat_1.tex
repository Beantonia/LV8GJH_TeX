\documentclass[aspectratio=169,12pt,fragile,allowframebreaks]{beamer}

\usepackage{amsthm}
\usepackage{hyperref}
\usepackage{xcolor}
\usepackage{hulipsum}
\usepackage{graphicx}

\usetheme{default}

\begin{document}
\transduration{10}
\author{Berecz Antónia} 
\title{Canciones} 
\subtitle{letras}  
\date{2022.11.03.} 
\maketitle 
\insertslidenumber.\ slide
\section{Első szakasz}
\begin{frame}{1. dia}{legelső dia}
Ez az első frame
\subsection{egy_pont_egy}
\hulipsum
\hulipsum
\end{frame}

\subsection{egy_pont_kettő}
\begin{columns}
\begin{column}{.5\linewidth}
első oszlop tartalma
\begin{enumerate}
\item egy
\item kettő
\item három
\item négy
\end{enumerate}
\begin{itemize}
\item<2-> egy
\item kettő
\item három
\item négy

\end{itemize}

\end{column}
\begin{column}{.5\linewidth}
második oszlop tartalma
\begin{figure}[bt]
\includegraphics[scale=0.2]{../kepek/kep}
\caption{képaláírás}
\end{figure}
\end{column}
\end{columns}

\section{Második szakasz}
\begin{frame}{2. dia}{második dia}
\pause
Igazán kreatív a cím
\subsection{block}
\begin{block}{block címe}
block tartalma
\end{block}
\pause
\subsection{exampleblock}
\begin{exampleblock}{block címe}
block tartalma
\end{exampleblock}
\subsection{alertblock}
\begin{alertblock}{block címe}
block tartalma
\end{alertblock}
\begin{alertblock}
block tartalma
\end{alertblock}
\end{frame}

\transdissolve<overlay>

\section{Harmadik szakasz}
\begin{frame}{3. dia}{harmadik dia}
\subsection{Maneskin}
\begin{theorem}[Maneskin]
"Tonight is gonna be the lonliest..."
\end{theorem}

\subsection{Claroscuro}
\begin{proof}[Claroscuro]
"Yo no tengo limitas solo el cielo..."
\end{proof}
\end{frame}

\section{Negyedik szakasz}
\begin{frame}{4. dia}{negyedik dia}
\begin{semiverbatim}
\\begin{itemize}
\\item \textcolor{red}{valami}
\\item \textcolor{red}{valami más}
\\begin{itemize}
\\item \textcolor{blue}{egyéb}
\\item \textcolor{blue}{valamicske}
\\end{itemize}
\\end{itemize}
\end{semiverbatim}
\end{frame}

\end{document}

