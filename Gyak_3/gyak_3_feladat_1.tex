\documentclass{article}
\usepackage[inline]{enumitem}
\usepackage{hulipsum}
\usepackage{float}
\usepackage{caption}
\usepackage{array}
\usepackage{colortbl}
\usepackage{wrapfig}
\usepackage{multirow}
\usepackage{graphicx}
\usepackage[table]{xcolor}

\begin{document}



\begin{itemize*}[itemjoin*={ ~és~ }]

\item[*]{fa}
\label{kerítés}
\item[*]{kert}
\item[*]{bokor}
\end{itemize*}

\begin{enumerate}
\item{fa}
\begin{enumerate}
\item[a.]{tölgy}
\begin{enumerate}
\item{kocsányos}
\label{kor}
\begin{enumerate}
\item{1 éves}
\end{enumerate}
\end{enumerate}
\end{enumerate}
\end{enumerate}
\hulipsum[1]

\begin{enumerate}[resume*]
\item virág
\item[*] bokor
\stepcounter{enumi}
\item kert
\end{enumerate}

\begin{description}[style=unboxed,style=nextline, style=sameline]
\item \hulipsum[1]
\item[\textsl{fák}] \hulipsum[1]
\item[\textsl{A fenyőerdők csodálatos élményekkel gazdagítják a túrázókat, emellett lakóhelyként is kitűnően funkcionálnak azok számára, akik a nagyvilágtól elvonulva szeretnék élni az életüket.}]  \hulipsum[1]
\end{description}

\ref{kerítés}
\ref{kor}

\ref{1.kép}
\ref{2.kép}

\newpage

\begin{figure}
\hulipsum[1]
\includegraphics[keepaspectratio,width=5cm, height=5cm]{demo}
\caption{1. ábra}
\label{1.kép}
\includegraphics[keepaspectratio,width=5cm, height=5cm,angle=180]{demo}
\caption{2. ábra}
\label{2.kép}
\hulipsum[1]
\end{figure}



\begin{tabular}{p{3em}||rcl|}
  & egy & kettő & három \\\hline\hline  Helló világ &  négy & öt & hat \\\cline{2-4}  & hét & nyolc & kilenc \\\cline{2-2}\cline{4-4} lórum ipse & tíz & & tizenkettő \\\hline
\end{tabular}

\bigbreak
\bigbreak

\rowcolors[\hline]{2}{blue!20}{yellow!20}
\begin{tabular}{r|c|l}
 egy & kettő & három \\\hline négy & öt & hat \\ hét & nyolc & kilenc \\ tíz & & tizenkettő \\
\end{tabular}

\bigbreak
\bigbreak

\rowcolors{2}{}{}
\begin{tabular}{r!{\color{red}\vline}c|l|}
egy & \multicolumn{2}{c|}{kettő} \\\hline 
\multirow{2}{2em}{négy} & öt & hat \\\cline{2-3}
  & \multicolumn{2}{c|}{\multirow{2}{2em}{nyolc}}   \\\cline{1-1} 
  tíz & \multicolumn{2}{c|}{}  \\\hline
\end{tabular}


\end{document}


