\documentclass{article}

\usepackage{float}
\usepackage{listings}
\usepackage{amsthm}
\usepackage{algorithmic}
\usepackage{hulipsum}
\usepackage{sidecap}
\usepackage{algorithm}


\newtheorem{tet}{Tétel}
\newtheorem{defin}{Definíció}
\newtheorem{lemma}[tet]{Lemma}
\newtheorem{feladat}{Feladat}[section]
\newtheorem*{megj}{Megjegyzés:}
\newfloat{forraskod}{htb}{for}
\floatstyle{ruled}
\newfloat{kod}{htb}{kod}



\begin{document}
\tableofcontents

\listof{forraskod}{Forráskódok listája:}
\listof{kod}{Kódok listája:}

\floatstyle{plain}
\begin{tet}[Arkhimédész]
Minden vízbe mártott test a súlyából annyit veszt, amennyi az általa kiszorított víz súlya.
\end{tet}

\floatstyle{plain}
\begin{tet}[Pitagorasz]
Derékszögű háromszög esetén az átfogó hosszának négyzete megegyezik az átfogók hosszának négyzetösszegével.
\end{tet}

\begin{proof}[Pitagorasz tétel bizonyítása]
Legyen a=3 cm, b=4cm,c=5cm
\end{proof}

\begin{defin}
A derékszögű háromszög...
\end{defin}


\begin{defin}
A négyzet olyan síkidom, amely...
\end{defin}

\begin{megj}
Ez itt egy megjegyzés
\end{megj}

\begin{lemma}
A paralelogramma két párhuzamos oldalpárral rendelkező négyszög.
\end{lemma}

\section{Könnyű feladatok}
\begin{feladat}
Az {an} számtani sorozat első és harmadik tagjának összege 26, a második és negyedik 
tagjának összege pedig 130.
\end{feladat}

\begin{feladat}
Hány fertály óra egy fertály nap, ha a fertály szó jelentése negyed?
\end{feladat}

\section{Haladó feladatok}
\begin{feladat}
A kapitány hajója 40 éves. Éppen kétszer annyi idős, mint amennyi a kapitány volt akkor, amikor a hajója annyi idős volt, mint most a kapitány. Hány éves a kapitány? 
\end{feladat}


\begin{forraskod}
\caption{ez egy kód}
\begin{verbatim}
\hulipsum
\section{első}
\end{verbatim}
\end{forraskod}

\hulipsum[3]

\begin{forraskod}
\caption{egy másik kód}
\begin{verbatim}
\begin{tet}[Pitagorasz]
Derékszögű háromszög esetén az átfogó hosszának négyzete megegyezik az átfogók hosszának négyzetösszegével.
\end{tet}
\end{verbatim}
\end{forraskod}

\begin{kod}
\caption{ez már tényleg egy kód}
\lstinputlisting[language=python,tabsize=4,numbers=left,stepnumber=4,frame=shadowbox]{code.txt}
\end{kod}




\end{document}